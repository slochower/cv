\documentclass[letterpaper,11pt]{article}
\usepackage{cv, booktabs, fontawesome}
\usepackage[scaled]{helvet}
\usepackage{graphicx, wrapfig}
\usepackage{chngpage}
\usepackage[export]{adjustbox}
\usepackage[bookmarks, colorlinks, breaklinks, 
pdftitle={Curriculum Viate: David R. Slochower},
pdfauthor={David R. Slochower}]{hyperref}  
\hypersetup{linkcolor=blue,
citecolor=blue,
filecolor=black,
urlcolor=blue} 
\usepackage{longtable}

% https://tex.stackexchange.com/questions/207760/add-an-image-to-each-entry-in-bibliography
% https://tex.stackexchange.com/questions/90031/patching-printbibliography-for-displaying-content-to-both-document-and-log
%\usepackage[sorting=ydnt,citestyle=authoryear,bibstyle=authoryear-comp,defernumbers=true,maxnames=20,giveninits=true, bibencoding=utf8, terseinits=true, uniquename=init,dashed=false,doi=false,isbn=false,natbib=true,url=false,backend=biber]{biblatex}
%\bibliography{my-publications}
%
%\DeclareSourcemap{
%  \maps[datatype=bibtex, overwrite]{
%    \map{
%      \step[fieldset=month, null]
%    }
%  }
%}
%\AtEveryBibitem{%
%  \clearlist{language}%
%}
%
% Nice!
% https://tex.stackexchange.com/questions/33330/make-one-authors-name-bold-every-time-it-shows-up-in-the-bibliography/211821#211821
%\newcommand{\makeauthorbold}[1]{%
%  \DeclareNameFormat{author}{%
%    \ifthenelse{\value{listcount}=1}
%    {%
%      {\expandafter\ifstrequal\expandafter{\namepartfamily}{#1}{\mkbibbold{\namepartfamily\addcomma\addspace \namepartgiveni}}{\namepartfamily\addcomma\addspace \namepartgiveni}}
%      %
%    }{\ifnumless{\value{listcount}}{\value{liststop}}
%        {\expandafter\ifstrequal\expandafter{\namepartfamily}{#1}{\mkbibbold{\addcomma\addspace \namepartfamily\addcomma\addspace \namepartgiveni}}{\addcomma\addspace \namepartfamily\addcomma\addspace \namepartgiveni}}
%        {\expandafter\ifstrequal\expandafter{\namepartfamily}{#1}{\mkbibbold{\addcomma\addspace \namepartfamily\addcomma\addspace \namepartgiveni\addcomma\isdot}}{\addcomma\addspace \namepartfamily\addcomma\addspace \namepartgiveni\addcomma\isdot}}%
%      }
%    \ifthenelse{\value{listcount}<\value{liststop}}
%    {\addcomma\space}
%  }
%}
%\makeauthorbold{Slochower}
% Use useprefix because of Ashton de Silva
%\ExecuteBibliographyOptions{useprefix=true}
%\renewcommand{\bibfont}{\normalfont\fontsize{10}{12.4}\sffamily}
%\usepackage{inconsolata}

%\makebibcategory{pprs}{Peer-reviewed publications}
%\makebibcategory{drafts}{Preprints and drafts}
%\makebibcategory{thesis}{Ph.D. thesis}

%\addtocategory{pprs}{
%slochower_motor-like_2018,
%yin_overview_2017,
%smith_salmon-derived_2016,
%slochower_physical_2015,
%slochower_counterion-mediated_2014,
%janmey_polyelectrolyte_2014,
%slochower_quantum_2013,
%moravcevic_kinase_2010,
%donald_transmembrane_2011,
%fujii_optimization_2012,
%wang_counterion-mediated_2014,
%yin_sampl5_2017,
%mobley_escaping_2018,
%bradley_molecular_2018,
%himmelstein_open_2018
%}
%
%\addtocategory{thesis}{
%slochower_multiscale_2014}
%
%
%\addtocategory{books}{
%slochower_lipid_2018}
%\nocite{*}
%\setlength{\bibitemsep}{1.8pt}
%\printbib{pprs}
%\printbib{books}
%\printbib{thesis}


% Trying something from John Chodera
\newcounter{articlenumber}
\setcounter{articlenumber}{1}
% NEW ARTICLE STYLE WITH IMAGE AND NO NUMBERS
\newcommand{\newarticle}[3]{
\begin{adjustwidth}{-0.05in}{-1in}  
\begin{tabular}{p{6in}p{0.9in}}
\parbox[c]{6in}{\arabic{articlenumber}. {#2} $\cdot$ {#3}} & \parbox[c]{1in}{\centering \includegraphics[height=0.8in]{thumbnails/{#1}}}
\end{tabular}
\end{adjustwidth}
\vspace{0.075in}
\addtocounter{articlenumber}{1}
}

\newcounter{booknumber}
\setcounter{booknumber}{1}
\newcommand{\newbook}[2]{
\begin{adjustwidth}{-0.05in}{-1in}  
\begin{tabular}{p{6in}p{0.9in}}
\parbox[c]{6in}{\arabic{booknumber}. {#2}} & \parbox[c]{1in}{\centering \includegraphics[height=0.8in]{thumbnails/{#1}}}
\end{tabular}
\end{adjustwidth}
\vspace{0.075in}
\addtocounter{booknumber}{1}
}

\newcommand{\newthesis}[2]{
\begin{adjustwidth}{-0.05in}{-1in}  
\begin{tabular}{p{6in}p{0.9in}}
\parbox[c]{6in}{{#2}} & \parbox[c]{1in}{\centering \includegraphics[height=0.8in]{thumbnails/{#1}}}
\end{tabular}
\end{adjustwidth}
\vspace{0.075in}
}

\newcounter{codenumber}
\setcounter{codenumber}{1}
\newcommand{\newcode}[4]{
\begin{adjustwidth}{-0.05in}{-1in}  
\begin{tabular}{p{6in}p{0.9in}}
\parbox[c]{6in}{\begin{tabular}{p{1.0in}p{0.6in}p{4in}}	#2 & #3 & #4 \end{tabular}} & \parbox[c]{1in}{\centering \includegraphics[width=0.8in]{thumbnails/{#1}}}
\end{tabular}
\end{adjustwidth}
\addtocounter{codenumber}{1}
}


\name{David R.~Slochower}
\info{\faicon{map-marker} & Skaggs School of Pharmacy and Pharmaceutical Sciences \\ &University of California, San Diego \\ &9500 Gilman Drive \#0736 \\ &La Jolla, CA, USA 92093\\
      \faicon{phone} & 610-639-4493\\
      \faicon{home} & www.slochower.name\\
      \faicon{envelope} & dslochower@ucsd.edu\\
      \faicon{twitter} & drslochower\\
      \faicon{github} & slochower}
      
\begin{document}

\maketitle
 
\section{Education and positions}
\begin{tabular}{lll}
2019- & \textbf{Open Force Field Postdoctoral Fellow}\\
             & Topic: Force Field Development \\
             \\
2015- & \textbf{Postdoctoral scholar}, University of California, San Diego\\
             & Advisor: Michael K.~Gilson, M.D., Ph.D.\\
             & Topic: Computational chemistry and biophysics\\
             \\
2014         & \textbf{Instructor}, University of Pennsylvania \\
             & Course: ``Molecular physiology and cellular engineering"\\
             \\
2007-2014 & \textbf{Ph.D. in Biochemistry and Molecular Biophysics}, University of Pennsvylvania \\
          & Advisor: Paul A.~Janmey, Ph.D. \\ 
          & Thesis: 
Multiscale simulations of phosphatidylinositol bisphosphate: understanding its \\ & biological role through physical chemistry \\
\\
2003-2007 & \textbf{A.B. \textit{cum laude} in Physics with distinction}, Kenyon College \\
          & Research: high energy nuclear imaging
\end{tabular}

\section{Research interests}
\begin{itemize}
    \item New methods for computing binding free energies
    \item Using computational chemistry to advance drug design  
    \item Improving force fields using open source science
    \item Nonequilibrium statistical mechanics of molecular motors

\end{itemize}

\nopagebreak[4]
\section{Previous research}
\begin{longtable}[H]{lp{16cm}}
2017- & \bf{Force field development with the Open Force Field Group} \\
      & \href{http://www.openforcefield.org}{Open Force Field Group} \\
2016- & \bf{Thermodynamics of host-guest molecular recognition} \\
      & Advisor: Michael K. Gilson, M.D., Ph.D. (University of California, San Diego)\\
2015- & \bf{Theory of molecular motors} \\
      & Advisor: Michael K. Gilson, M.D., Ph.D. (University of California, San Diego)\\
2015-2016 & \bf{Inhibitors of prion protein} \\
      & Advisors: Michael K. Gilson, M.D., Ph.D. and Christina Sigurdson, D.V.M, Ph.D. (University of California, San Diego)\\
2014-2015 & \bf{Simulations and docking of macrocycles} \\
     & Advisors: Ravi Radhakrishnan, Ph.D. (University of Pennsylvania) and Mark A. Lemmon, Ph.D. (Yale University) \\
2009-2014 & \bf{Quantum, all-atom, and coarse-grained molecular dynamics of membranes} \\
     & Advisor: Paul A. Janmey, Ph.D. (University of Pennsylvania) \\
2008 & \bf{Simulations of viral entry into cells} \\
     & Advisors: William DeGrado, Ph.D. (University of California, San Francisco) and Michael L. Klein, Ph.D. (Temple University) \\
2007 & \bf{Experimental single molecule biophysics} \\
     & Advisor: Yale E. Goldman, M.D., Ph.D. (University of Pennsylvania) \\
2007 & \bf{Computational design of synthetic peptides} \\
     & Advisor: Jeffery Saven, Ph.D. (University of Pennsylvania) \\
2006-2008 & \bf{Coded aperture imaging} \\
     & Advisors: John Idoine, Ph.D. (Kenyon College), John Frangioni, M.D., Ph.D. (Harvard University), and Richard Lanza, Ph.D. (Massachusetts Institute of Technology) \\
2005 & \bf{Analysis of protein hydration shells in simulations} \\
     & Advisor: Matthias Buck, Ph.D. (Case Western Reserve University) \\
\end{longtable}

%\section{Experience}
%\begin{tabular}{lp{16cm}}
%2014-2015 & \textbf{Postdoctoral fellow, Radhakrishnan and Lemmon Labs}, University of Pennsylvania\\

%        & I worked on clustering protein ensembles from simulations of receptor tyrosine kinases to improve small molecule docking. \\
%        
%2009-2014 & \textbf{Graduate student, Janmey Lab}, University of Pennsylvania\\ 
%         & In my thesis work, I ran multiscale molecular dynamics simulations to model the
%structure and dynamics of lipids in the plasma membrane under
%physiological conditions. I used this model to determine the
%interactions between lipids and intracellular counterions as well as
%what factors affect the ability of lipid clusters to form in the
%plasma membrane. This work is in collaboration with the Radhakrishnan
%lab.\\
%
%Summer 2006 & \textbf{Summer student, Frangioni Lab}, Harvard University\\ &
%In the summer of 2006, I was responsible for continuing the collaboration between Prof. John
%Frangioni at Harvard, Prof. John Idoine at Kenyon, and Prof. Dick
%Lanza at MIT on a coded aperture nuclear imaging device that
%integrates multiple medical imaging modalities. I developed a three
%dimensional iterative reconstruction algorithm that was used to image
%small animals injected with 99mTc coupled to a bone growth marker.\\
%            
%Summer 2005 & \textbf{Summer student, Buck Lab}, Case Western Reserve University\\ &
%During summer 2005, I analyzed two supercomputer simulations of
%lysozyme surrounded by varying amounts of water molecules to
%characterize protein hydration shells. The dynamics of waters were
%measured over 50 ns while the motion of the protein side chains were
%compared to NMR data.\\
%
%2005-2007 & \textbf{Undergraduate student, Idoine Lab}, Kenyon College\\ 
%          & From September
%2005 to May 2007, I worked in the lab of Prof. John Idoine on nuclear
%imaging and its application to medicine. Over these years, I learned
%how to operate and repair a high-energy NaI(Tl) gamma ray detector,
%managed the Solaris workstations, and designed new reconstruction
%algorithms and software. This research was supplemented by two
%independent study courses Prof. Idoine and I designed.
%
%\end{tabular}

\section{Preprints and working manuscripts}
\newarticle{cyclodextrin.png}{
{\bf Slochower DR}, Henriksen NM, Chodera JD, Mobley DL, Gilson MK. "Binding thermodynamics of host-guest systems with SMIRNOFF99Frosst from the Open Force Field Group"}{
\href{https://slochower.github.io/smirnoff-host-guest-manuscript/}{Manuscript} 
$\cdot$
\href{https://github.com/slochower/smirnoff-host-guest-simulations}{GitHub  (analysis code)} $\cdot$
\href{https://github.com/slochower/smirnoff-host-guest}{GitHub  (parameterization code)}
}
%
\newarticle{sampling.png}{
Rizzi A, Jensen T, {\bf Slochower DR}, Aldeghi M, Gapsys V, Bosisio S, Henriksen NM, de Groot BL, Dickson A, Michel J, Gilson MK, Shirts MR, Mobley DL, Chodera JD. "The SAMPL6 SAMPLing challenge:  Assessing the reliability and efficiency of binding free energy calculations"}{
\href{https://github.com/mobleylab/sampl6}{GitHub  (challenge information)} $\cdot$
\href{https://github.com/MobleyLab/SAMPL6/tree/update-sampling/host_guest/Analysis/SAMPLing/Data}{GitHub  (results)}
}
%
\section{Peer-reviewed publications}
\newarticle{manubot.png}{
Himmelstein DS, {\bf Slochower DR}, Malladi VS, Greene CS, Gitter, A. "Open Collaborative Writing with Manubot" {\it Accepted with minor revisions at PLOS Computational Biology} 2019
}
{
\href{https://github.com/manubot/manubot}{GitHub  (code)} $\cdot$
\href{https://github.com/greenelab/meta-review}{GitHub (manuscript)} $\cdot$
\href{http://blogs.nature.com/naturejobs/2018/02/20/techblog-manubot-powers-a-crowdsourced-deep-learning-review/}{Nature TechBlog}
}
%
\newarticle{bilayers}{
Bradley RP$^*$, {\bf Slochower DR$^*$}, Janmey PA, Radhakrishan R. "Molecular modeling of divalent cation-specific nano-clusters of phosphoinositides in physiologically-composed bilayers" {\it Under Review at JACS} 2018
}
{
\href{https://github.com/biophyscode}{GitHub}
}
%
\newarticle{smirnoff.png}{
Mobley DL, Bannan CC, Bayly CI, Rizzi A, Chodera JD, Lim VT, Lim NM, Beauchamp KA, {\bf Slochower DR}, Shirts MR, Gilson MK, Eastman PK. "Escaping Atom Types in Force Fields Using Direct Chemical Perception" {\it Journal of Chemical Theory and Computation}, 14 6706-6092, 2018}
{\href{https://github.com/openforcefield}{GitHub}}
%
\newarticle{motors.png}{
{\bf Slochower DR}, Gilson MK. "Motor-like Properties of Nonmotor Enzymes" {\it Biophysical Journal} 114:9, 2018}
{\href{https://www.biorxiv.org/content/early/2018/02/07/121848}{bioRxiv} $\cdot$ \href{http://dx.doi.org/10.1016/j.bpj.2018.02.008}{DOI} $\cdot$ \href{https://github.com/GilsonLabUCSD/nonequilibrium}{GitHub} $\cdot$ \href{https://www.cell.com/biophysj/fulltext/S0006-3495\%2818\%2930444-2}{New and Notable} $\cdot$ \href{http://ucsdhealthsciences.tumblr.com/post/173707350285/its-not-intelligent-design-so-how-did}{UCSD In the News}
}
%
\newarticle{apr.png}{
Yin J, Henriksen NM, {\bf Slochower DR}, Gilson MK. "The SAMPL5 host-guest challenge: computing binding free energies and enthalpies from explicit solvent simulations by the attach-pull-release (APR) method" {\it Journal of Computer-Aided Molecular Design} 1:31 133-145, 2017}
{\href{http://dx.doi.org/10.1007/s10822-016-9970-8}{DOI}}

%
\newarticle{sampl5.png}{
Yin J, Henriksen NM, {\bf Slochower DR}, Shirts MR, Chiu MW, Mobley DL, Gilson MK. "Overview of the SAMPL5 host-guest challenge: Are we doing better?" {\it Journal of Computer-Aided Molecular Design} 1:31 1-19, 2017}
{\href{http://dx.doi.org/10.1007/s10822-016-9974-4}{DOI} $\cdot$ 
\href{https://github.com/GilsonLabUCSD/SAMPL5-bootstrapping-error-analysis}{GitHub}}
%
\newarticle{thrombin.png}{
Smith JR, Galie PA, {\bf Slochower DR}, Weisshaar CL, Janmey PA, Winkelstein BA. "Salmon-derived thrombin inhibits development of chronic pain through an endothelial barrier protective mechanism dependent on APC" {\it Biomaterials} 80 96-105, 2016}
{\href{http://dx.doi.org/10.1016/j.biomaterials.2015.11.062}{DOI}}
%
\newarticle{pccp}{
{\bf Slochower DR}, Wang Y-H, Radhakrishnan R, Janmey PA. "Physical chemistry and membrane properties of two phosphatidylinositol bisphosphate isomers" {\it Physical Chemistry Chemical Physics } 17:19 12608-12615, 2015}
{\href{http://dx.doi.org/10.1039/C5CP00862J}{DOI}}
%
\newarticle{clusters}{
Wang Y-H, {\bf Slochower DR}, Janmey PA. "Counterion-mediated cluster formation by polyphosphoinositides" {\it Chemistry and Physics of Lipids } 182 38-51, 2014}
{\href{http://dx.doi.org/10.1016/j.chemphyslip.2014.01.001}{DOI}}
%
\newarticle{pattern}{
{\bf Slochower DR}, Wang Y-H, Tourdot RW, Radhakrishnan R, Janmey PA. "Counterion-mediated pattern formation in membranes containing anionic lipids" {\it Advances in Colloid and Interface Science } 208 177-188, 2014}
{\href{http://dx.doi.org/10.1016/j.cis.2014.01.016}{DOI}}
%
\newarticle{pf1}{
Janmey PA, {\bf Slochower DR}, Wang Y-H, Wen Q, Ceber A. "Polyelectrolyte properties of filamentous biopolymers and their consequences in biological fluids" {\it Soft Matter} 10:10 1439-1449, 2014}
{\href{http://dx.doi.org/10.1039/c3sm50854d}{DOI}}
%
\newarticle{quantum}{
{\bf Slochower DR}, Huwe PJ, Radhakrishnan R, Janmey PA. "Quantum and All-Atom Molecular Dynamics Simulations of Protonation and Divalent Ion Binding to Phosphatidylinositol 4,5-Bisphosphate (PIP$_2$)" {\it The Journal of Physical Chemistry B} 117:28 8322-8329, 2013}
{\href{http://dx.doi.org/10.1021/jp401414y}{DOI}}
%
\newarticle{ca}{
Fujii, H, Idoine JD, Gioux S, Accorsi R, {\bf Slochower DR}, Lanza R, Frangioni JV. "Optimization of Coded Aperture Radioscintigraphy for Sentinel Lymph Node Mapping" {\it Molecular Imaging and Biology} 14:2 173-182, 2012}
{\href{http://dx.doi.org/10.1007/s11307-011-0494-2}{DOI}}
%
\newarticle{transmembrane}{
Donald JE, Zhang Y, Fiorin G, Carnevale V, {\bf Slochower DR}, Gai F, Klein ML, DeGrado WF. "Transmembrane orientation and possible role of the fusogenic peptide from parainfluenza virus 5 (PIV5) in promoting fusion" {\it Proceedings of the National Academy of Sciences} 108:10 3958-3963, 2011}
{\href{http://dx.doi.org/10.1073/pnas.1019668108}{DOI}}
%
\newarticle{ka1}{
Moravcevic K, Mendrola JM, Schmitz KR, Wang Y-H, {\bf Slochower DR}, Janmey PA, Lemmon MA. "Kinase Associated-1 Domains Drive MARK/PAR1 Kinases to Membrane Targets by Binding Acidic Phospholipids" {\it Cell} 143:6 966-977, 2010}
{\href{http://dx.doi.org/10.1016/j.cell.2010.11.028}{DOI}}
%
%
$^*$ These authors contributed equally.
\section{Book chapters}
\newbook{shape}{
{\bf Slochower DR}, Wang Y-H, Radhakrishan R, Janmey PA. "Lipid membrane shape evolution and the actin cytoskeleton" in {\it Handbook of Lipid Membranes, Molecular and Materials Aspects}, Eds. Safinya C, R\"{a}dler, J. (2018)}

\section{Ph.D. thesis}
\newthesis{thesis}{{\bf Slochower DR}. "Multiscale simulations of phosphatidylinositol bisphosphate: understanding its biological role through physical chemistry" {\it University of Pennsylvania}, 2014}

\section{Software packages authored and co-authored}

    \newcode{pAPRika}{\texttt{pAPRika}}{\href{https://github.com/slochower/pAPRika}{GitHub}}{Free energy calculations with AMBER and OpenMM}
    %
    \newcode{speakeasy}{\texttt{speakeasy}}{\href{https://github.com/nhenriksen/speakeasy}{GitHub}}{Automates the conversion of SMIRNOFF parameters to AMBER force field files}
    %
    \newcode{taproom}{\texttt{taproom}}{\href{https://github.com/slochower/host-guest-benchmarks}{GitHub}}{Host-guest (and soon, protein-ligand) benchmark systems for free energy calculations, with support for SMIRNOFF parameter coverage}
    %
    \newcode{smirnovert}{\texttt{smirnovert}}{\href{https://github.com/slochower/smirnoff-host-guest}{GitHub}}{Convert host-guest systems into SMIRNOFF force fields}
    %
    \newcode{manubot-logo}{\texttt{manubot}}{\href{https://github.com/greenelab/manubot}{GitHub}}{Automated scholarly publishing}
    %
    \newcode{biophyscode}{\texttt{BioPhysCode}}{\href{https://github.com/biophyscode}{GitHub}}{Tools for building and analyzing membrane simulations}

\section{Software skills}
\begin{itemize}
    \item Proficient programming in Python (extensive use of notebooks) with knowledge of Bash, C++, FORTRAN, Perl, and R.
    \item Active user of GitHub and continuous integration platforms; experienced writing unit, integration, and regression tests.
    \item Expert with molecular dynamics simulations using AMBER and OpenMM; familiar with Gromacs, CHARMM, and LAMMPS. 
    \item Experience working with Schrödinger and Chemical Computing Group software suites.
    \item Accomplished using OpenEye Toolkits; familiar with KNIME and RDKit.
\end{itemize}


\section{Invited talks, posters, abstracts, and workshops (selected)}
\begin{tabular}{ll p{14 cm}}
2019 & Talk & \bf{Benchmarking emerging force fields from the Open Force Field Initiative} \\
&& 2019 American Chemical Society meeting \\
2019 & Talk & \bf{Binding free energy calculations using the attach-pull-release method} \\
&& 2019 AMBER developers' meeting \\
2019 & Talk & \bf{Benchmarking emerging force fields with binding thermodynamics calculations} \\
&& 2019 University of California Chemical Symposium \\

% 2018 & Talk & \bf{Using calorimetric data to drive accuracy in computer-aided drug design} \\
% && Presented by Michael K. Gilson at The North American Calorimetry Conference \\
2017 & Talk & \textbf{Directional motion in chiral molecules out of equilibrium} \\
     && 253rd American Chemical Society Meeting \\
% 2017 & Talk & \textbf{Are all enzymes molecular motors? An effect of binding and catalysis out of equilibrium} \\
%     && Presented by Michael K. Gilson at the 254th American Chemical Society Meeting \\
2017 & Poster & \textbf{Directional and driven motion in enzymes out of equilibrium} \\
     && 61st Annual Biophysical Society Meeting \\     
2014 & Talk & \textbf{Multiscale modeling of polyphosphoinositides}\\
     && University of California, San Diego\\
2014 & Talk & \textbf{Physical chemisty of phosphatidylinositol isomers}\\
     && University of California, Irvine\\
2013 & Talk & \textbf{Membranes: Polyphosphoinositides}\\
     && Friday Research Discussions, University of Pennsylvania\\
2013 & Poster & \textbf{Quantum and All-atom Molecular Dynamics Simulations of Proton   Binding to Phosphatidylinositol 4,5-bisphosphate (PIP$_2$)}\\
     && 57th Biophysical Society Meeting\\
% 2012 & Poster & \textbf{Multiscale modeling of membrane curvature induced by epsin} \\
%     && Presented by Ryan Bradley at the 244th American Chemical Society Meeting \\
2012 & Talk & \textbf{Molecular Dynamics Simulations of Ion Binding and Protonation of 
Phosphatidylinositol Bisphosphate (PIP$_2$)} \\
     && 244th American Chemical Society Meeting \\
2012 & Talk & \textbf{Simulations of membrane electrostatics with PtdInsP$_2$}\\
     && George W. Raiziss 30th Annual Retreat \\
2012 & Poster & \textbf{Molecular Dynamics Simulations of Phosphatidylinositol Bisphosphate (PIP$_2$)}\\
     && American Physical Society, March Meeting\\
2011 & Talk & \textbf{Molecular Dynamics Simulations of Membranes}\\
     && 47th New England Complex Fluids Workshop \\
% 2011 & Poster & \textbf{Association of transmembrane helices in viral fusion peptides suggests a protein-centric mechanism of membrane fusion} \\
%     && Presented by Giacomo Fiorin at the 55th Biophysical Society Meeting \\
2011 & Poster & \textbf{Molecular Dynamics Simulations of Monolayers and Membranes with Phosphatidylinositol Bisphosphate} \\
     && 55th Biophysical Society Meeting\\
% 2011 & Workshop & \textbf{Demsond Workshops} \\
%     && D.~E.~Shaw Research\\
% 2011 & Workshop & \textbf{Temple High Performance Computing (HPC)}\\
%     && Axel Kohlmeyer, Temple University\\
2010 & Talk & \textbf{Simulating highly charged monolayers}\\
     && Mechanistic Studies in Membrane Biophysics: Experiments and Theory, 
Telluride Science and Research Workshop\\
2010 & Poster & \textbf{Simulations of Monolayers with Phosphatidylinositol Bisphosphate}\\
     && Gotham-Metro Condensed Matter Meeting\\
% 2010 & Poster & \textbf{Viral fusogenic peptides form transmembrane helical bundles: Implications for the mechanism of fusion}\\
%     && Presented by Vincenzo Carnevale at the 239th American Chemical Society\\
%2010-2013 && Various posters at the Dr.~George W. Raiziss Retreat \\
%    && Department of Biochemistry and Biophysics, University of Pennsylvania
%2007 & \textbf{What are the questions that drive evolution?} (Talk)\\
%     & Biology Department, Kenyon College\\
%2006 & \textbf{Coded Aperture Imaging} (Talk)\\
%     & Senior exercise, Kenyon College\\
%2005 & \textbf{The dynamics of water surrounding proteins} (Talk)\\
%     & Physics Colloquium, Kenyon College
\end{tabular}

\section{Awards and grants}
\begin{tabular}{ll}
2019 & Open Force Field Initiative Postdoctoral Fellow \\
2012-2013 & NIH T32 Structural Biology Training Grant\\
2011 & Juan Grana Graduate Teaching Assistantship\\
2010-2012 & NIH T32 Interdisciplinary Cardiovascular Training Grant\\
2007 & Distinction in Physics (best research), Kenyon College\\
2007 & Sigma Xi, The Scientific Honor Society\\
2004-2007 & Dean's List, Kenyon College\\
2005 & Best Summer Project (biophysics), Case Western Reserve University\\
2001 & Science Olympiad, National Champion Team
\end{tabular}

\section{Teaching and mentoring experience}
\begin{tabular}{lp{16cm}}
Fall 2014 & Molecular Physiology \& Cellular Engineering -- University of Pennsylvania\\ & 
I was responsible for creating the syllabus, giving lectures, designing project assignments, 
and grading for one-half of this course.\\

Spring 2011 & Macromolecular Biophysics II -- University of
Pennsylvania\\ & I was in charge of arranging lectures, holding office
hours and regular review sessions, grading homework and exams for
first and second year graduate students.\\ 
2009- & Mentored high school, undergraduate, and graduate students in research. \\
Spring 2005 & Programming I
-- Kenyon College\\ & I designed regression tests for weekly project
assignments, graded, and then posted my own solutions to the class.
\end{tabular}

\section{Service}
\begin{itemize}
    \item Reviewer for \emph{Soft Matter}
    \item Reviewer for \emph{European Biophysics Journal}
    \item Reviewer for \emph{Scientific Reports}
    \item Reviewer for \emph{Nature Structural \& Molecular Biology}
    \item Member, Biophysical Society
    \item Member, American Chemical Society
\end{itemize}

\section{References}
Michael K.~Gilson, M.D., Ph.D. \\
Professor and Chair in Computer-Aided Drug Design \\
Co-Director UC San Diego Center for Drug Discovery Innovation \\
Skaggs School of Pharmacy and Pharmaceutical Sciences \\
University of California, San Diego \\
9500 Gilman Drive \#0736 \\
La Jolla, CA 92093 \\
Voice: (858) 822-0622 \\
\url{mgilson@ucsd.edu}
\vskip 0.2 in
Paul A.~Janmey, Ph.D. \\
Professor of Physiology \\
Associate Director, Institute for Medicine and Engineering \\
University of Pennsylvania \\
1010 Vagelos Research Laboratories \\
3340 Smith Walk \\
Philadelphia, PA 19104 \\
Voice: (215) 573-7380 \\
\url{janmey@mail.med.upenn.edu}
\vskip 0.2 in
Ravi Radhakrishnan, Ph.D. \\
Professor of Bioengineering \& Chemical and Biomolecular Engineering \\
Department of Bioengineering \\
University of Pennsylvania \\
210 S. 33rd Street \\
240 Skirkanich Hall \\
Philadelphia, PA 19104 \\
Voice: (215) 898-0592 \\
\url{rradhak@seas.upenn.edu}
\vskip 0.2in
David L.~Mobley, Ph.D. \\
Professor of Pharmaceutical Sciences and Chemistry \\
University of California, Irvine \\
3134B Natural Sciences 1 \\
Mail Code: 3958 \\
Irvine, CA 92697 \\
Voice: (949) 824-6383 \\
\url{dmobley@uci.edu}
\end{document}

